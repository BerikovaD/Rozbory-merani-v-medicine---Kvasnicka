% ================================================================
\section{Prehľad}
% ================================================================
\begin{frame}{Čo sa naučíte v tomto kurze}
	\begin{block}{1. Používať R}
		Trocha neskromné, nemáme veľa času.
		\begin{description}
			\item[Základy R] aké nástroje používať, základnú syntax, kde hľadať pomoc
			\item[Načítať dáta] import z textových súborov, Excelu, web stránok
			\item[Preskúmať dáta] kreslenie - \texttt{ggplot2}
			\item[Upraviť dáta] manipulácia s dátami, súhrny, filtrovanie atď. - \texttt{tidyverse}
			\item[Analyzovať dáta] lineárny model, ANOVA atď.
			\item[Validovať analýzu] simulácia, replikácia (bootstrap)
		\end{description}
		Nie presne v tomto poradí, budeme sa hýbať v kruhoch.
	\end{block}
\end{frame}

\begin{frame}{Čo sa naučíte v tomto kurze}
	\begin{block}{2. Používať moderný ekosystém pre prácu s dátami}
		\begin{description}
			\item[Open source] software, nepotrebujeme nakupovať drahý štatistický software, ani MS Office, všetko čo potrebujeme sa dá stiahnuť z Internetu.
			\item[Version control] Chceme, aby sa dáta dali zdieľať a boli chránené pred stratou alebo nechcenou zmenou.
			\item[Rôzne zdroje dát] Chceme pracovať s dátami z viacerých možných zdrojov - textové súbory, Excel, JSON, databázy atď.
			\item[Zdieľanie dát a analýzy] Chceme, aby si ľudia mohli skontrolovať našu analýzu (a my chceme skontrolovať tú ich!)
		\end{description}
	\end{block}
\end{frame}


% ================================================================
% Outline
% ================================================================

\begin{frame}{Outline}
\tableofcontents
\end{frame}

% ================================================================
\section{Čo potrebujete pre tento kurz}
% ================================================================

% ----------------------------------------------------------------
\subsection{Znalosti}
% ----------------------------------------------------------------
\begin{frame}{Čo sa očakáva, že budete vedieť}
	\begin{block}{Základné znalosti z pravdepodobnosti a štatistiky}
		\begin{itemize}
			\item Stačí, aby ste sa veľmi nezľakli, keď poviem t-test.
			\item Máte určitú prax v spracovaní a zobrazovaní dát.
		\end{itemize}
	\end{block}
	\begin{block}{Programovanie}
		\begin{itemize}
			\item Očakávam, že máte za sebou kurz programovania, v hocičom.
			\item Napríklad keď idete písať kód, začnete tým, že si zapnete anglickú klávesnicu.
			\item A pochytili ste trocha algoritmického myslenia. 
			\item Budeme sa učiť nový jazyk a používať nové nástroje, takže pôjdeme od nuly.
		\end{itemize}
	\end{block}
	\begin{block}{Angličtina}
		\begin{itemize}
			\item Je mi ľúto, ale bez angličtiny budete mať v tomto kurze vážne ťažkosti.
			\item Predovšetkým si veľmi ťažko budete hľadať pomoc na Internete, a to je prvá vec, ktorú človek robí, keď mu niečo nefunguje alebo nevie, ako niečo urobiť. 
		\end{itemize}
	\end{block}
\end{frame}

% ----------------------------------------------------------------
\subsection{Laptop}
% ----------------------------------------------------------------
\begin{frame}{Ešte potrebujete laptop}
	\begin{block}{Laptop}
		\begin{itemize}
			\item Windows 7 alebo 10, alebo Linux
			\begin{itemize}
				\item Windows 10 má WSL - Windows Subsystem for Linux - a umožňuje vám lepšie používať niektoré veci, napríklad \texttt{git}.
				\item Ale Linux je na to ešte lepší.
			\end{itemize}
			\item Nepotrebujete mať extra silný procesor alebo veľa pamäti, aspoň nie pre tento kurz.
			\item Ani nepotrebujete mať veľa miesta na disku (stačí gigabajt).
		\end{itemize}
	\end{block}
\end{frame}

% ----------------------------------------------------------------
\subsection{Software}
% ----------------------------------------------------------------

\begin{frame}
	\begin{block}{Nainštalovať software}
		\begin{itemize}
			\item Potrebujete mať nainštalované R a RStudio.
			\item Zriaďte si účet na GitHub (\texttt{https://www.github.org}) a stiahnite si \emph{GitHub Desktop}. Ak máte Linux, stačí vám nainštalovať git. 
			\begin{itemize}
				\item Toto stačí v priebehu kurzu, svoj GitHub account budete potrebovať na odovzdanie zadaní.
				\item Aj keď sa zdá, že to nesúvisí so štatistikou, používanie git-u může byř jedna z najdôležitejších vecí, ktoré sa tu naučíte, preto toto neodkladajte. 
			\end{itemize}
			\item Návod na inštaláciu R/RStudio a na zriadenie GitHub konta nájdete v GitHub repozitári tohoto kurzu, \texttt{https://www.github.com/PKvasnick/RTutorial/}. Ak by ste mali problémy s inštaláciou, rýchlo sa ozvite (peter.kvasnicka\@mff.cuni.cz)
		\end{itemize}
	\end{block}
\end{frame}

% ----------------------------------------------------------------
\subsection{Informácie}
% ----------------------------------------------------------------
\begin{frame}{Kde hľadať informácie}
	\begin{block}{Príručky}
		Príručiek je veľa, väčšina aktuálnych a moderných je v angličtime.
		\begin{itemize}
			\item Na Pinterestovej stránke \texttt{https://sk.pinterest.com/peterkvasnika/my\_r/} nájdete odkazy na niekoľko internetových portálov a PDF dokumentov, ktoré vám môžu pomôcť v začiatkoch.
			\begin{itemize}
				\item Niekoľko z nich je slovenčine/češtine.
			\end{itemize}
			\item Na portáli CRAN (Comprehensive R-Archive Network - \texttt{https://cran.r-project.org/}) nájdete prehľad dokumentácie k R.
		\end{itemize}
	\end{block}
	\begin{block}{Nápoveď v R a RStudio}
		\begin{itemize}
			\item R má svoj vlastný help systém, naučíte sa s ním pracovať.
			\item RStudio má takisto svoje helpy.
		\end{itemize}
	\end{block}
\end{frame}

\begin{frame}{Kde hľadať informácie}
	\begin{block}{Internet}
		\begin{itemize}
			\item To čo programátor robí najčastejšie je, že vysvetlí Googlu lámanou angličtinou čo chce urobiť (\texttt{R create dataframe}), alebo priamo do riadku vyhľadávača skopíruje chybovú hlášku. 
			\item S vysokou pravdepodobnosťou nájdete použiteľnú odpoveď, či už je vaša otázka triviálna alebo zložitá.
			\begin{itemize}
				\item Tú odpoveď nájdete najčastejšie na webe StackOverflow, \texttt{https://stackoverflow.com}, s ktorým sa určite spriatelíte.
			\end{itemize}
			\item Časom prídete na to, že kúsok fungujúceho kódu býva užitočnejší ako podrobný výklad syntaxe. 
		\end{itemize}
	\end{block}
\end{frame}

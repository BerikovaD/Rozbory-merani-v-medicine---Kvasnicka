\documentclass[9pt]{beamer}
\usetheme{Pittsburgh}
\usepackage[utf8]{inputenc}
\usepackage[slovak]{babel}
\usepackage{amsmath}
\usepackage{amsfonts}
\usepackage{amssymb}
\usepackage{graphicx}
\usepackage{booktabs}

\usepackage{caption}
\usepackage{subcaption}

\author{Peter Kvasnička}
\title{Úvod do R}
%\setbeamercovered{transparent} 
%\setbeamertemplate{navigation symbols}{} 
%\logo{} 
\institute{Univerzita Karlova, Praha} 
\date{Kurz pre 4. ročník BMF \newline Jeseň 2017} 
\subject{Statistics and data science} 

\beamertemplatenavigationsymbolsempty

\setbeamerfont{page number in head/foot}{size=\large}
\setbeamertemplate{footline}[frame number]

\begin{document}

\begin{frame}
\titlepage
\end{frame}

% ================================================================
\section{Úvod}
% ================================================================
\begin{frame}{Čo sa naučíte v tomto kurze}
	\begin{block}{1. Používať R}
		Trocha neskromné, nemáme veľa času.
		\begin{description}
			\item[Načítať dáta] import z textových súborov a Excelu
			\item[Preskúmať dáta] kreslenie - \texttt{ggplot2}
			\item[Upraviť dáta] manipulácia s dátami, súhrny, filtrovanie atď. - \texttt{tidyverse}
			\item[Analyzovať dáta] lineárny model, ANOVA atď.
			\item[Validovať analýzu] simulácia, replikácia (bootstrap)
		\end{description}
		Nie presne v tomto poradí, budeme sa hýbať v kruhoch.
	\end{block}
\end{frame}

\begin{frame}{Čo sa naučíte v tomto kurze}
	\begin{block}{2. Používať moderný ekosystém pre prácu s dátami}
		\begin{description}
			\item[Open source] software, nepotrebujeme nakupovať drahý štatistický software, ani MS Office, všetko čo potrebujeme sa dá stiahnuť z Internetu.
			\item[Version control] Chceme, aby sa dáta dali zdieľať a boli chránené pred stratou alebo nechcenou zmenou.
			\item[Rôzne zdroje dát] Chceme pracovať s dátami z viacerých možných zdrojov - textové súbory, Excel, JSON, databázy atď.
			\item[Zdieľanie dát a analýzy] Chceme, aby si ľudia mohli skontrolovať našu analýzu.
		\end{description}
	\end{block}
\end{frame}


% ================================================================
% Outline
% ================================================================

\begin{frame}{Outline}
\tableofcontents
\end{frame}

% ================================================================
\section{Čo potrebujete pre tento kurz}
% ================================================================

% ----------------------------------------------------------------
\subsection{Znalosti}
% ----------------------------------------------------------------
\begin{frame}{Čo sa očakáva, že budete vedieť}
	\begin{block}{Základné znalosti z pravdepodobnosti a štatistiky}
		\begin{itemize}
			\item Stačí, aby ste sa veľmi nezľakli, keď poviem t-test.
			\item Máte určitú prax v spracovaní a zobrazovaní dát.
		\end{itemize}
	\end{block}
	\begin{block}{Programovanie}
		\begin{itemize}
			\item Očakávam, že máte za sebou kurz programovania, v hocičom.
			\item Napríklad keď idete písať kód, začnete tým, že si zapnete anglickú klávesnicu.
			\item A pochytili ste trocha algoritmického myslenia. 
			\item Budeme sa učiť nový jazyk a používať nové nástroje, takže pôjdeme od nuly.
		\end{itemize}
	\end{block}
	\begin{block}{Angličtina}
		\begin{itemize}
			\item Je mi ľúto, ale bez angličtiny budete mať v tomto kurze ťažkosti.
			\item Predovšetkým si veľmi ťažko budete hľadať pomoc na Internete, a to je prvá vec, ktorú človek robí, keď mu niečo nefunguje alebo nevie, ako niečo urobiť. 
		\end{itemize}
	\end{block}
\end{frame}

% ----------------------------------------------------------------
\subsection{Laptop}
% ----------------------------------------------------------------
\begin{frame}{Ešte potrebujete laptop}
	\begin{block}{Laptop}
		\begin{itemize}
			\item Windows 7 alebo 10, alebo Linux
			\begin{itemize}
				\item Windows 10 má WSL - Windows Subsystem for Linux - a umožňuje vám lepšie používať niektoré veci, napríklad \texttt{git}.
				\item Ale Linux je na to ešte lepší.
			\end{itemize}
			\item Nepotrebujete mať extra silný procesor alebo veľa pamäti, aspoň nie pre tento kurz.
		\end{itemize}
	\end{block}
	\begin{block}{Inštalovaný software}
		\begin{itemize}
			\item Potrebujete mať nainštalované R a RStudio.
			\item Zriaďte si účet na GitHub (\texttt{https://www.github.org}) a stiahnite si \emph{GitHub Desktop}. Ak máte Linux, stačí vám nainštalovať git. Toto stačí v priebehu kurzu, svoj GitHub account budete potrebovať na odovzdanie zadaní.
			\item Návod na inštaláciu nájdete v GitHub repozitári tohoto kurzu, \texttt{https://www.github.com/PKvasnick/RTutorial/}
			\item Ak by ste mali problémy s inštaláciou, rýchlo sa ozvite.
		\end{itemize}
	\end{block}
\end{frame}


% ================================================================
\section{Conclusions}
% ================================================================

\begin{frame}{Conclusions}
	\begin{itemize}
		\item The SVD timing estimation is still not implemented due to SVD re-factoring activities.
		\item Most of the stuff will be implemented within 10 days.
		\item The overall functionality depends on the availability of calibration data.
		\item SVD digitizer urgently needs re-factoring, as it is slow and does signal time-dependence wrong.
	\end{itemize}
\end{frame}

\end{document}
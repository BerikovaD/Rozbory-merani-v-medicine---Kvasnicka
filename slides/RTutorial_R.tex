% ================================================================
\section{R}
% ================================================================
\begin{frame}{Prečo R?}
	\begin{block}{Máme predsa ...}
		\begin{description}
			\item[Excel] a iné tabuľkové programy
			\item[SPSS] Statisticu, Minitab a iné komerčné programy poskytujúce analýzu na kľúč
		\end{description}
		\alert{Tak prečo mám používať niečo, čo sa treba určitý čas učiť?}
	\end{block}
\end{frame}

\begin{frame}{Niekoľko dôvodov}
	\begin{block}{Excel nie je štatistický program}
		\begin{itemize}
			\item Excel je výborný nástroj na vkladanie dát, získavanie a konsolidáciu dát z databáz a na základné úpravy dát
			\item Ale nie je dobrý na výmenu dát (polo-proprietárny formát - nikdy nevieme, kedy sa zmení)
			\item Vzorce v bunkách sa ťažko spravujú a neexistuje praktický spôsob, ako nezávisle dokumentovať, čo sa ako počíta.
			\item Nemáme výstrahu, ak náhodne zmeníme obsah bunky
			\item Nástroje pre štatistiku sú implementované ledabylo.
			\item Grafy sú na zaplakanie.
		\end{itemize}
	\end{block}
\end{frame}

\begin{frame}{Niekoľko dôvodov}
	\begin{block}{Robíme stále zložitejšie analýzy}
		\begin{itemize}
			\item Chceme skúmať analyzovať zložité a veľké dáta
			\item Chceme validovať našu analýzu pomocou simulácií a replikácie - potrebujeme analýzy opakovať tisíckrát
			\item Chceme formulovať a testovať zložité modely (\emph{Data Science})
		\end{itemize}
	\end{block}
	\begin{block}{Chceme zdieľať dáta a analýzu}
		\begin{itemize}
			\item Potrebujeme otvorený software a nie drahé štatistické balíky alebo MS Office
			\item Potrebujeme otvorené formáty dát
			\item Chceme software, ktorý sa \emph{rýchlo inovuje}
			\item Chceme software, ktorý je správny
		\end{itemize}
	\end{block}
\end{frame}

\begin{frame}{Preto chceme R!}
	\begin{block}{R je programovate2né}
		\begin{itemize}
			\item R je interpretovaný programovací jazyk
			\item R podporuje integráciu s inými programovacími jazykmi - môžeme volať funkcie naprogramované v C++, Fortrane ap., čo podstatne kompenzuje pomalosť vlastného interpreta R.
			\item R spolupracuje s Pythonom, Javou a ďalšími jazykmi, ktoré používajú vývojári v data science
			\item R má výborné IDE, RStudio, a najnovšie aj Visual Studio!
		\end{itemize}
	\end{block}
	\begin{block}{R má bohaté rozhrania pre dáta}
		\begin{itemize}
			\item R umožňuje čítať dáta z veľkého množstva vstupných formátov:
			\begin{itemize}
				\item textových súborov
				\item Excelu
				\item JSON
				\item databáz
				\item Apache Spark-u
				\item \dots
			\end{itemize}
			\item R dokáže dáta, grafy a reporty exportovať do veľkého množstva formátov.
		\end{itemize}
	\end{block}
\end{frame}

\begin{frame}{Preto chceme R!}
	\begin{block}{R je rozšíriteľné}
		\begin{itemize}
			\item To, čo robí R skutočne cenným, je ekosystém rozšírení - baličkov (packages)
			\item Tieto baličky obsahujú všetky štatistické metódy, ktoré kedy budete potrebovať
			\item Balíčky sídlia na serveri CRAN, a môžete si ich ľahko doinštalovať cez interpret R (\texttt{install.packages(<menobalíčka>}
			\item Balíčky neustále pribúdajú: Ak niekto opublikuje novú štatistickú metódu, s veľkou pravdepodobnosťou ju hneď implementuje v R.
		\end{itemize}
	\end{block}
	\begin{block}{R je renomované a spoľahlivé}
		\begin{itemize}
			\item Pretože R používa veľa ľudí, je dobre otestované a všetky prípadné chyby sú hneď odstránené.
			\item Ak si svoje dáta analyzujete v R, nikto sa nebude pýtať, či ste správne počítali ANOVu.
			\item Kód vašej analýzy je univerzálne zrozumiteľný doklad o tom, čo ste robili.
		\end{itemize}
	\end{block}
\end{frame}

\begin{frame}{Nemá chybu...?}
	\begin{block}{R má svoje špecifiká a slabé stránky}
		\begin{itemize}
			\item R sa pôvodne vyvinulo z funkcionálneho a objektovo-orientovaného jazyka S. Preto niektoré veci pracujú trocha odlišne.
			\item Pretože R má za sebou dlhú históriu, obsahuje niekoľko súperiacich koncepcií a funkčných rozhraní. Preto niektoré veci možno robiť rôznymi spôsobmi, a naopak niektoré podobné veci musíte robiť odlišne.
			\item R je pomalé. Treba sa vyhýbať zložitým programovým konštrukciám v R (cyklom \texttt{for} a podobne), a používať čo najviac metafunkcie R (\texttt{apply}), aby sa počítanie robilo v C a Fortrane, a nie v R. 
			\item Napriek tomuto všetkému sa základy programovania v R možno naučiť pomerne rýchlo a pomerne rýchlo získať výsledky.
		\end{itemize}
	\end{block}
\end{frame}
